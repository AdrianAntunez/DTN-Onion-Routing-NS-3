\documentclass[runningheads,a4paper]{llncs}

\usepackage[american]{babel}
\usepackage[utf8]{inputenc}

\usepackage{svg}
\usepackage{amsmath}

%extended enumerate, such as \begin{compactenum}
\usepackage{paralist}
\usepackage{color}
\usepackage{graphicx}

\usepackage{caption}
\usepackage{subfig}
\usepackage{algorithm}
\usepackage[noend]{algpseudocode}
\renewcommand{\algorithmicforall}{\textbf{for each}}
\algnewcommand\algorithmicinherit{\textbf{INHERIT:}}
\algnewcommand\INHERIT{\item[\algorithmicinherit]}
\algnewcommand\algorithmicinput{\textbf{INPUT:}}
\algnewcommand\INPUT{\item[\algorithmicinput]}
\algnewcommand\algorithmicoutput{\textbf{OUTPUT:}}
\algnewcommand\OUTPUT{\item[\algorithmicoutput]}
\makeatletter
\def\BState{\State\hskip-\ALG@thistlm}
\makeatother
\newcommand{\red}[1]{\textcolor{red}{#1}}

%put figures inside a text
%\usepackage{picins}
%use
%\piccaptioninside
%\piccaption{...}
%\parpic[r]{\includegraphics ...}
%Text...

%Sorts the citations in the brackets
%\usepackage{cite}

%for easy quotations: \enquote{text}
\usepackage{csquotes}

\usepackage[T1]{fontenc}

%enable margin kerning
\usepackage{microtype}

%better font, similar to the default springer font
\usepackage[%
rm={oldstyle=false,proportional=true},%
sf={oldstyle=false,proportional=true},%
tt={oldstyle=false,proportional=true,variable=true},%
qt=false%
]{cfr-lm}
%
%if more space is needed, exchange cfr-lm by mathptmx
%\usepackage{mathptmx}

%for demonstration purposes only
\usepackage[math]{blindtext}

\usepackage[
%pdfauthor={},
%pdfsubject={},
%pdftitle={},
%pdfkeywords={},
bookmarks=false,
breaklinks=true,
colorlinks=true,
linkcolor=black,
citecolor=black,
urlcolor=black,
%pdfstartpage=19,
pdfpagelayout=SinglePage
]{hyperref}
%enables correct jumping to figures when referencing
\usepackage[all]{hypcap}

\usepackage[capitalise,nameinlink]{cleveref}
%Nice formats for \cref
\crefname{section}{Sect.}{Sect.}
\Crefname{section}{Section}{Sections}
\crefname{figure}{Fig.}{Fig.}
\Crefname{figure}{Figure}{Figures}

\usepackage{xspace}
%\newcommand{\eg}{e.\,g.\xspace}
%\newcommand{\ie}{i.\,e.\xspace}
\newcommand{\eg}{e.\,g.,\ }
\newcommand{\ie}{i.\,e.,\ }

%introduce \powerset - hint by http://matheplanet.com/matheplanet/nuke/html/viewtopic.php?topic=136492&post_id=997377
\DeclareFontFamily{U}{MnSymbolC}{}
\DeclareSymbolFont{MnSyC}{U}{MnSymbolC}{m}{n}
\DeclareFontShape{U}{MnSymbolC}{m}{n}{
    <-6>  MnSymbolC5
   <6-7>  MnSymbolC6
   <7-8>  MnSymbolC7
   <8-9>  MnSymbolC8
   <9-10> MnSymbolC9
  <10-12> MnSymbolC10
  <12->   MnSymbolC12%
}{}
\DeclareMathSymbol{\powerset}{\mathord}{MnSyC}{180}

% correct bad hyphenation here
\hyphenation{op-tical net-works semi-conduc-tor}

\begin{document}

%Works on MiKTeX only
%hint by http://goemonx.blogspot.de/2012/01/pdflatex-ligaturen-und-copynpaste.html
%also http://tex.stackexchange.com/questions/4397/make-ligatures-in-linux-libertine-copyable-and-searchable
%This allows a copy'n'paste of the text from the paper
\input glyphtounicode.tex
\pdfgentounicode=1

\title{Onion Routing in Predictable Delay Tolerant Networks}
%If Title is too long, use \titlerunning
%\titlerunning{Short Title}

%Single insitute
\author{Adrian Antunez-Veas \and Guillermo Navarro-Arribas}
%If there are too many authors, use \authorrunning
%\authorrunning{First Author et al.}
\institute{Department of Information and Communications Engineering, \\
				Universitat Autònoma de Barcelona (UAB), 08193 Cerdanyola, Spain\\
				\email{\{aantunez, gnavarro\}@deic.uab.cat}}

%Multiple insitutes
%Currently disabled
%
\iffalse
%Multiple institutes are typeset as follows:
\author{Firstname Lastname\inst{1} \and Firstname Lastname\inst{2} }
%If there are too many authors, use \authorrunning
%\authorrunning{First Author et al.}

\institute{
Insitute 1\\
\email{...}\and
Insitute 2\\
\email{...}
}
\fi
			
\maketitle

\begin{abstract}
With the growth of Internet of Things (IoT) the data as well as the anonymity preservation has become a challenging research topic. Using Delay Tolerant Network (DTN) with onion routing an alternative secure network that hides the source identity can be achieved. We consider as predictable DTN at such networks where the behaviour is known in advance or where a repetitive action occurs over the time like in public transport networks. This project shows how the prior stage of path choosing in the onion routing can be achieved using the information provided by predictable networks while the security of our proposed method is analysed.
\end{abstract}

\keywords{Onion Routing, Delay-tolerant Network, Anonymous networking, Privacy, Security}

\section{Introduction}\label{sec:intro}

Delay Tolerant Networks (DTNs) \cite{dtn-definition} are a class of networks aimed to provide end-to-end communication into environments lacking continuous connectivity or with long delays. To do so, a new network layer called bundle is used. It works on top of the transport layer to handle disruptions and long delays using the store-carry-and-forward principle. The received data, called bundle, is stored in persistent storage on each DTN node, forwarding it when a new opportunity of contact appears.

There are several scenarios where DTNs can be useful like sensing systems, deep space missions, disaster recovery, and rural area networks. Many of these applications need to preserve user information as they can share sensitive information. In traditional encryption the transmitted data is hidden from third parties.  Nonetheless, the source and destination identities are shown, leading to potential information leaks about sender and receiver. Techniques can be used to preserve the anonymity too, e.g: in anonymous sensing systems a user may want to preserve his identity to avoid possible reprisal.

A popular technique to achieve anonymous communication in traditional networks is onion routing \cite{onion-routing}. Onion routing provides source anonymity by encapsulating messages in layers of encryption. It is said that onion routing is not aplicable to DTNs because onion routing method needs to know the route in advance to perform the cryptographic layers for each node \cite{dtn-security-analysis}. Nevertheless, this protocol can be applicable in deterministic (predictable) DTNs. Deterministic DTNs have been studied before \cite{probabilistic-dtn}, \cite{deterministic-dtn}. Using predictable DTNs, the route to encrypt the message needed in onion routing can be obtained. At the same time, we solve the challenging routing problem in DTNs \cite{oracle-types} because we decide to which nodes forward the message.

Previous research work on the use of onion routing in DTNs is very scarce. Possibly, the most relevant is ARDEN \cite{arden}. ARDEN uses groups of nodes to perform the layering process and broadcast messages between these groups. The message's route is chosen randomly. To perform the cryptographic layers attribute based encryption (ABE) is used. Our method is easier to implement than ARDEN, and does not need complex protocol cryptography methods like ABE.

The remainder of the paper is organized as follows: Section \ref{sec:relevant-concepts} presents a deeper review of relevant concepts used in our proposal. Section \ref{sec:proposal} shows the process to use onion routing over predictable DTNs. A method to know the route in advance, given a set of restrictions, for the layering process is proposed. Section \ref{sec:sanalysis} presents an informal analysis of the security of our proposal from the point of view of passive and active attackers. Section \ref{sec:evaluation} presents the results of our simulation over a public transport network (considered predictable networks). This results show the performance of our method in terms of different statistical metrics. Section \ref{sec:conclusions} provides concluding remarks as well as possible future research work.

\section{Related work}\label{sec:rwork}

\red{Aim of this section: Show the existing research about this topic. Identify his flaws and explain briefly what we do to solve (or improve) them.}

\section{Proposal}\label{sec:proposal}

\red{Proposal goes here}.

\section{Security analysis}\label{sec:sanalysis}

%\red{Aim of this section: •.}

In this section, we evaluate our proposed scheme from the aspect of security. First, we define the threat model, defining important considerations regarding the scenario under study. After, we define different kinds of adversaries explaining what they can do and what we do to preserve the sender anonymity as well as the privacy of exchanged information.

\subsection{Threat model}

Alice (\textit{A}) wants to communicate to Bob (\textit{B}) without revealing information about her. Using onion routing we intend to improve A's anonymity as well as data privacy against adversaries that can be passive,i.e: eavesdropping messages or take an active paper in this scenario, i.e: make modifications or attack to other nodes of the network. In general, security threats can be divided into passive and active threats. We consider that nodes in our scenario use strong cryptographic algorithms with enough key lengths to prevent practical cryptanalysis attacks to discover the source, the destination or the contents of a message. 

\subsection{Passive adversaries}

Passive attacks are those that perform guessing simply observing user traffic patterns from "passive" nodes. 

As is explained in \cite{latency-leak} if the attacker is the destination of the message, he can learn something from the delay between messages.  This kind of attacks does not work when sending start time, known as \textit{t} parameter in our path choosing method, is not highly predictable \cite{enpassant}.

Another attacker model is a set of compromised nodes that works together to retrieve information leading to break the users privacy. We have two different situations to deal with: the multiple decryption and the sending node periodicity. First, the attacker will be able to decrypt more layers, or messages if one of the nodes is the destination, because they have their corresponding private keys. Second, there are scenarios where a node or a set of them rarely transmit information to others, discarding this nodes from the probable sending set, a globally passive adversary can correctly guess the source of a message. To overcome such attacks the \textit{n} value can be increased, i.e: the number of nodes that has a single path to send the message. As much nodes, much layers a message will have. There is a trade-off between efficiency and security deciding \textit{n} value. To solve the guessing issue the creation of dummy packets when ingress throughput drops below a certain threshold \cite{arden} may help to prevent such attacks.

Is important to note that an attacker can combine previously explained attacks to increase their chance of guessing.

To decrease the probability of guess the path, different paths are retrieved using our path choosing method and one of them is chosen randomly.

\subsection{Active adversaries}

Active adversaries are those that performs actions against other nodes or modifies information that cross through them. As is the case of passive nodes, an attacker controlling a single node will be unable to extract the source of a message because of the use of multiple layers of encryption \cite{arden}. There are several possible attacks to do against onion routing by malign node in the network \cite{congestion-attack}, \cite{location-attack}, \cite{latency-leak}.

An attacker who has a control of a node of the network can attacking non-observed nodes to shut them down. Prohibiting the communication between the source and the destination if that node was implied in involved in the chosen path. This kind of attacks are called Denial of Service DoS attacks and can be addressed improving the robustness of the nodes as well as with reputation systems as in discussed in conclusions and future work section \ref{sec:conclusions}.

Message modifications by attackers are easily detected using cryptographic hash methods. Other attacks like masquerading (nodes pretending to be different nodes) are solved as layers of encryption check the node identity. The key management process is safely done in a prior stage.

\section{Evaluation}\label{sec:evaluation}

\red{Evaluation goes here}.

\section{Conclusions and future work}\label{sec:conclusions}

\red{Aim of this section: There are two parts: \\ The first one the conclusions of this project. What we can extract from our research. Could be useful? Important things that needs to be remarked, etc. \\ The second one the future work. What needs to be done in the future to solve some undesired behaviours, explore unresearched lines of this work, etc.}

\red{[Conclusions part]}



\red{[Future work]}

\red{Search and analyse efficient ways of path choosing, being able to use the algorithm in resource-constrained computers. }

\red{In the network could exist black holes, i.e: nodes that drop all the traffic. In order to mitigate the impact of such nodes in the whole network a reputation system could be used. By this way the reputation value will be shared among all the nodes in the network. The path choosing algorithm should be modified too to take this value into consideration in the choosing process. Finally this will lead into another security analysis due to the more reputation a node has, most probable is to be one of the chosen. }

\red{The simulation model could be adapted to consider traffic modifications, generating the enough information to decrease the number of  failed contacts due to a bad contact prediction.}

\subsubsection*{Acknowledgements}

\red{Acknowledgements go here}.

%%%%%%%%%%%%%%%%%%%%%%%%%%%%%%%%%%%%%%%%%%%%%%%%%%

\bibliography{paper}

All links were last followed on \today.

\bibliographystyle{ieeetr}

\end{document}
