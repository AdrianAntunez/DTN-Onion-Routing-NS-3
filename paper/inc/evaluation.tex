\section{Evaluation}\label{sec:evaluation}
%\red{Aim of this section: •.}

\red{Aim of this section: Brief introduction of what are we going to talk about in this section.}

In this section the security of our proposal is evaluated using a realistic scenario providing information about the devices and other parameters used in the evaluation.

\subsection{Scenario: Campus buses}

\red{Aim of this section: Explain a bit how the campus scenario works and how can be this useful in practice.}

In order to test our proposal we considered a very little public transportation network that works inside the Autnonomous University of Barcelona (UAB) composed by 5 buses that makes different routes around the UAB campus.

Each bus has a DTN node achieving secret communications as well as source anonymity using this network. There are several applications that can take profit of such networks like anonymous reporting systems.

\subsection{Mobility Model}

\red{Aim of this section: explain how we get this scenario: open street maps -> sumo -> ns-3... }

We obtained the mobility model going along different stages. First, we exported the UAB Campus map from OpenStreetMaps into SUMO \red{\cite{sumo}} software, filtering some unnecessary items with the Java OpenStreetMap editor \red{\cite{josm}} tool.

% bus-schedule: http://www.uab.cat/doc/horaris_busUAB_2015
Once the campus roads was imported in SUMO, we recreated the bus movements of each bus taking into consideration the official bus schedule of the UAB public transportation network \red{\cite{bus-schedule}}. In addition, we tuned some bus characteristics like acceleration and deceleration parameters in order to get coherent travel times.

%sumo-to-ns2: http://www.ijarcsse.com/docs/papers/Volume_4/4_April2014/V4I4-0416.pdf
Finally, we exported the model to a the NS-2 mobility trace as is explained in \red{\cite{sumo-to-ns2}}. The NS-2 mobility trace can be used with the well-known Network Simulator NS-3. We used the simulator to obtain important contact related data of the campus network, i.e: information about the duration of the contacts as well as the instant of time when they occurred.

\subsection{Simulation setup}

\red{Aim of this section: Explain and define the values used in the simulation itself as well as how we know that there is a neighbour able to contact with.}

\subsection{Simulation results}

\red{Aim of this section: Explain the results of this simulation. What we get and explaining why.}